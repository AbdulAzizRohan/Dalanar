\documentclass[12pt,a4paper]{article}

% Required packages
\usepackage[top=30mm,bottom=25mm,left=30mm,right=25mm]{geometry}
\usepackage{fontspec}
\setmainfont{Times New Roman}
\usepackage{setspace}        % For line spacing
\usepackage{graphicx}        % For images
\usepackage{amsmath}         % For mathematical equations
\usepackage{hyperref}
\usepackage[capitalise]{cleveref} % For smart references
\usepackage{caption}         % For figure and table captions
\usepackage{fancyhdr}        % For header and footer
\usepackage{titlesec}        % For formatting titles
\usepackage{enumitem}        % For list customization
\usepackage{longtable}
\usepackage{subcaption}
\usepackage[version=4]{mhchem}
\usepackage{float}
\usepackage{bookmark}
\usepackage{calc}            % For width calculations
\usepackage{svg}
\usepackage[breakable, skins]{tcolorbox}
\usepackage{listings}
\usepackage{xcolor}
\usepackage{minted}
\usepackage{pdfpages}
\usepackage[backend=biber, style=ieee, sorting=none]{biblatex}
\addbibresource{ref.bib} %Import the bibliography file

\renewcommand{\figurename}{Fig.}
\renewcommand{\tablename}{Table} 
\def\figureautorefname{Fig.}      
\def\tableautorefname{Table}    

\definecolor{codegreen}{rgb}{0,0.6,0}
\definecolor{codegray}{rgb}{0.5,0.5,0.5}
\definecolor{codepurple}{rgb}{0.58,0,0.82}
\definecolor{backcolour}{rgb}{0.95,0.95,0.92}

\lstdefinestyle{mystyle}{
    backgroundcolor=\color{backcolour},   
    commentstyle=\color{codegreen},
    keywordstyle=\color{magenta},
    numberstyle=\tiny\color{codegray},
    stringstyle=\color{codepurple},
    basicstyle=\ttfamily\footnotesize,
    breakatwhitespace=false,         
    breaklines=true,                 
    captionpos=b,                    
    keepspaces=true,                 
    numbers=left,                    
    numbersep=5pt,                  
    showspaces=false,                
    showstringspaces=false,
    showtabs=false,                  
    tabsize=2
}

\lstset{style=mystyle}

\fancypagestyle{mainstyle}{
    \fancyhf{} % Clear all header/footer
    \fancyhead[R]{\textbf{Page | }\thepage} % Right header with custom format
    \renewcommand{\headrulewidth}{0pt} % Remove header line
}

\pagestyle{mainstyle}

% Document begin
\begin{document}

% Title page
\begin{titlepage}
  \begin{center}
    \vspace*{1cm}

    \Large
    \textbf{Fire Alarm System with Data Logging Capabilities}\\

    \vspace{0.5cm}
    \normalsize
    \textbf{Course Code: ESE 4126}\\
    \textbf{Course Name: Sessional on ESE 4125}

    \vspace{2cm}
    \large
    \textbf{Lab Report}

    \vspace{3cm}
    \normalsize
    \textbf{Submitted by:}\\
    Mahmud Abdul Aziz Rohan\\
    Roll No: 2013021\\ 
    Year: 4\textsuperscript{th}    Term: 1\textsuperscript{st}\\
    Department of Energy Science and Engineering\\
    Khulna University of Engineering \& Technology

    \vspace{2cm}
    \textbf{Submitted to:}\\
    Dr. Md. Hasan Ali \\
    Professor \\ 
    \&\\
    Mostafizur Rahaman \\
    Lecturer\\
    Department of Energy Science and Engineering\\
    Khulna University of Engineering \& Technology\\
    Khulna-9203, Bangladesh \\

    \vspace{2cm}
    \textbf{Date of Submission: \today}
  \end{center}
\end{titlepage}   

% Main content begins
\clearpage
\pagenumbering{arabic}

\section{Introduction}

\par
Fire incidents represent a pervasive and perishing threat to both human life and
property across residential, commercial, and industrial settings
\parencite{Suwarjono2021}. The primary factor to initiate a fast response and
reduce damage requires immediate and dependable detection of fire events
\parencite{Siregar2024}. Traditional fire detection systems often lack the
real-time monitoring and advanced diagnostic capabilities that modern embedded
systems can provide \parencite{Sarwar2019}. The combination of inexpensive
dependable microcontrollers with sensor systems enables the creation of
cost-effective fire safety solutions which deliver both accessibility and
operational efficiency \parencite{Sarwar2019}.

\par
This project aims to design an Intelligent Fire Alarm System utilizing the
Arduino Uno microcontroller, a widely recognized and easily programmable
platform based on the ATmega328P \parencite{Sarwar2019}. The system includes
multiple fire detection sensors, such as smoke detectors (MQ-2/MQ-7) and
temperature sensors (e.g., TMP36), to monitor the environment for signs of a
potential fire.

\par
A critical feature of this project is its data logging capability. The system's
ability to both capture and save sensor information with event timestamps
delivers superior benefits when compared to conventional alarm systems
\parencite{Wilson2020}. Data logging enables permanent system observation which
supports environmental condition analysis before alarm activation and helps
decrease false alarms by tracking sensor data patterns over time
\parencite{Sarwar2019}. The collected data serves multiple purposes such as
system diagnostics and threshold optimization and post-incident analysis.

\par
The primary objective of this report is to detail the design, construction, and
testing of the fire alarm system, demonstrating the effective integration of the
Arduino Uno, sensor components, and a data logging mechanism to create a
reliable, low-cost safety solution. 

\section{Objectives}

Fire alarm systems are essential safety equipments. Making them capable of
logging data will enable safety officers to analyze the fire hazard and risk in
any particular environment for a specific period of time and will allow them to
make predictions about them. Keeping this in context, the objectives of this
experiment are
  
\begin{enumerate}[label=\roman*]
  \item to design a fire alarm system using Arduino UNO in Tinkercad
  \item to incorporate data logging methods with the fire alarm system
  \item to extract the logged data from the system 
\end{enumerate}

\section{Fire Alarm System with Data Logging Capabilities}

\subsection{Arduino UNO R3}

\begin{figure}[H]
  \begin{center}
    \includegraphics[width=0.8\textwidth]{figures/arduino.pdf}
  \end{center}
  \caption{Arduino UNO R3}\label{fig:arduino}
\end{figure}


\subsection{System Description}

\begin{figure}[H]
  \begin{center}
    \includegraphics[width=0.95\textwidth]{figures/diagram.png}
  \end{center}
  \caption{Diagram of the setup in Tinkercad}\label{fig:diagram}
\end{figure}



\begin{figure}[H]
  \begin{center}
    \includegraphics[width=0.95\textwidth]{figures/schematic.pdf}
  \end{center}
  \caption{Schematic diagram of the circuit used in the project}\label{fig:schematic}
\end{figure}



\subsection{Code}

\lstinputlisting[caption=sketch.ino]{code/sketch.ino}

\par 
The \texttt{setup()} function is used for initilizing the input and output
ports. The \texttt{loop()} function runs with a delay of 1 s declared in the
last line of the function as \texttt{delay(1000)}.
The conditions for the alart signals are either the gas sensor readings should
be above 400 or the temperature reading should be above 30 degree celsius or
both. 
The \texttt{logData(unsigned long timestamp)} function is used for logging the
data to the serial monitor. From this monitor the logged output can be copied
and used as a CSV file for further analysis. The time lag between each data logging is
set at 5000 ms is the begining of the code as a global variable \texttt{const
unsigned long logInterval}. 

\section{Result}

\begin{figure}[H]
  \begin{center}
    \includegraphics[width=0.95\textwidth]{figures/result_normal.png}
  \end{center}
  \caption{Fire Alarm System under normal conditions}\label{fig:result_normal}
\end{figure}


\begin{figure}[H]
  \begin{center}
    \includegraphics[width=0.95\textwidth]{figures/result_alert_gas.png}
  \end{center}
  \caption{Alarm engaged because of excess amount of gas}\label{fig:result_alert_gas}
\end{figure}


\begin{figure}[H]
  \begin{center}
    \includegraphics[width=0.95\textwidth]{figures/result_alert_temp.png}
  \end{center}
  \caption{Alarm engaged because of high temperature}\label{fig:result_alert_temp}
\end{figure}


\begin{figure}[H]
  \begin{center}
    \includegraphics[width=0.95\textwidth]{figures/result_data_logging.png}
  \end{center}
  \caption{Logged data in the serial monitor in CSV format}\label{fig:result_data_logging}
\end{figure}



\section{Discussion}

The results show that the fire alarm system is functional in the simulated
conditions. Figure~\ref{fig:result_normal} is showing the normal conditions
when the alarm is not yet engaged. In Figure~\ref{fig:result_alert_gas} and
Figure~\ref{fig:result_alert_temp}, the alarm system is engaged because of
excess gas and high temperature. This confirms the expected behaviour of the
system under actual conditions. The data from the alarm system can be collected
from the serial monitor as shown in Figure~\ref{fig:result_data_logging}. This
data can be anayzed using various tools for future predictions of abnormal
conditions, or detecting anomaly in the system.

\section{Conclusion}

A fire alarm system was successfully designed in Tinkercad using Arduino UNO R3.
Data logging was also incorporated using the serial print strategy. The data can
be extracted from the serial monitor by copying from the monitor and storing it
in a file. Further modifications can be made by using SD card module to store
the data in a file. 

\printbibliography

\end{document}
